\chapter*{Abstract}
The term document classification is not imaginative for the current era as numerous people have worked on it concerning their problem and made their lives easier. Since the digital environment has become a solid trend in every field and industry that's why a lot of communities are actively working on problems related to document classification. Almost every other firm has to deal with a huge amount of documents in terms of assets. If one has to process or classify these documents manually then it's not only a tedious task to do but also we have to keep doing it for every new document which is not the most fun exercise for anyone. So the organized structure of documents and it's quick access is very significant for them. That is where automatic document classification comes into play and that is what the target of this thesis.
\newline
\newline
Furthermore, we are targeting some specific documents which are used while tax filing. The primary goal of the thesis is to take the tax documents as input and predict/classify the type of tax document. To solve this problem, we tried and used Deep Learning approach and came up with three different solutions which we will talk about in detail in Chapter \ref{chap:5}. We used the Convolutional Neural Network (CNN) as our deep learning model and used both Sequential and transfer learning models VGG19 \cite{vgg_19} for different solutions. We will also compare and analyze how these solutions performing. In the dataset used to train the model, the possible types of documents are income statement, salary slips, and general receipts. Since these documents are confidential and are not publically available, so we had to generate the documents with different layouts by ourselves. We all know the significance of the quality and the quantity of the dataset to obtain a well-trained model, so a lot of efforts have been made for this purpose. We will have a detailed look on data generation process in Chapter \ref{chap:3}.



