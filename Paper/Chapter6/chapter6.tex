\par
In the final chapter, we will briefly summarize the goals, implementations, and limitation of the thesis. Also, we will describe the ideas for future work and what are the possible ways to enhance the solution and achieve maximum results.
\section{Conclusion}
We started this thesis by considering the current trend of digitalization in the world and aims to obsolete the manual environment by automating the domain of document management and classification. This thesis particularly targets the specific types of documents that we can use to claim our tax and aims to classify these tax documents automatically. As of now, as per our research, there is no system available to classify the German tax documents. Due to which it was also very difficult to gather the data for our problem. Somehow, we managed to generate the data which was not in as much quantity that it should be due to which we had to narrow down the classes for our document types. In the further section, we will briefly discuss the presented solutions, compare them and see what the limitations of solutions are. Also, we will see whether our goal is achieved or not.
\subsection{Proposed Solution and Comparison}
The goal of the thesis was to find a solution that can reduce the human effort and time spending on classifying and labeling the tax documents manually and provide a system that can perform this task automatically. Although our main solution is based on deep learning techniques, there was no intention to specifically choose that to do the job, rather the purpose was to find the solution irrespective of the technology stack. That is why we presented different solutions based on simple algorithms as well as a deep learning methodology.
\newline
\par
Our first solution is based on the simple scoring algorithm that analyzes textual information from the documents to determine the document type. It maintains a dictionary of relevant words with weighted scores which it uses during the classification. For the classification, it accepts the input in the form of the text string and calculates the score by analyzing each word in the text for all the document types and returns the document type with the highest score. 
\newline
\par
In the second solution, we are using a machine learning technique by analyzing the textual information extracted from the documents which can be in the form of pdf or images. We apply some pre-processing to the data to make it as relevant as possible and then convert the data into a vector array to train the model and make the predictions.
\newline
\par
In the third and major solution, we are using a transfer learning methodology to develop a model for predicting the tax documents. In this solution, instead of using textual information, we are using visual information to train the model.
\newline
\par
All the above solutions are targeting to classify the tax documents by applying various approaches. Although we are achieving the same accuracy of 100\% in each solution, but if we compare the results, the solutions that are relying on textual information are performing better and accurate than the solution which uses visual information. However, this is because of the reason for lack of data. Obviously, in a text-based solution, we don't need much data because every other document for a particular document type will contain more or less the same kind of information. Hence, we are getting good results in text-based solution but if we increase the dataset for visual data, hopefully, we can achieve better results here as well.
\subsection{Goals Achieved?}
The goal of the thesis was to develop an intelligent system to classify German tax documents. Our actual scope of the classes for tax documents were income tax statements, salary slips, general receipts, disability certificates, auxiliary costs receipts, and unemployment certificate. Out of this, we provided the solution that can classify income tax statements, salary slips, and general receipts for now. However, in this thesis, we managed to come up with an approach which we can use in the long run to expand the scope of our system by adding more dataset for the remaining document types. That was also one of the goals of the thesis to develop a scalable architecture and provide such provisions that allow the end-users to add future document types without much changing in the systems. For this purpose, we designed the system that provides different APIs to add future document types as new classes, to add dataset for any particular document type and to re-train the model after adding the dataset for any existing class or a new class. This will save a lot of time as it reduces the manual efforts of configuring the project locally and export the model when a new training occurs, every time to the live system.
\subsection{Limitations}
As mentioned in the previous section, as of now, our system is only capable of classifying limited types of documents which are income tax statements, salary slips, and general receipts and cannot classify the rest of the types. This happens because the kind of data that we are dealing with in this problem is not general-purpose and considered as quite confidential. Due to which we couldn't find much data on the internet and other open-source datasets like Kaggle, ImageNet, etc. Another limitation is that there are some document types in which the format or template is not fixed and can exist in several formats. For example, if we talk about salary slips, each company has its own format or template to generate the salary slips for their employees. So, in the case of visual data, we cannot rely on limited formats and have to add as many formats as possible during the training of the model. This scenario belongs to salary slips and general receipts as both of them can exist in multiple formats. For now, we have gathered limited formats for these two types of documents and trained our model on these formats. So, it is fairly possible that the third solution, which uses visual information to predict the document type, will not work on any other format which appears totally different and was not present during the training.
\section{Future Work}
Here, we will discuss the possibilities to optimize or enhance the provided solutions, what can be the possible efforts to do so and what are the things required to do in the future. For now, we have come up with few actions that need to be taken in the future after the thesis.
\subsection{Dataset Gathering}
As we know that we narrowed down the classes for our classification system due to the lack of quality data. However, in the future, we are aiming to gather the data for the rest of the document types as well as further data for the existing document types in our system but in different formats, because it is very necessary to add more formats to generalize the model for these types of documents. The possible approach for further data gathering for salary slip documents is to get in touch with different companies to ask for their formats whereas, for general receipts since it is not that confidential, we can use our personal expense receipts and ask our friends and colleagues for them. For the rest of the types, we will follow the same approach along with searching on the internet or by generating the data by ourselves.
\subsection{Interface for the APIs}
In this thesis, we have designed the system in such a way that we are providing provisions to the end-users by exposing APIs for different purposes like adding dataset, adding new class, and to train the model. At least these are the most basic operation that we can think of for now that we would need in the future quite frequently. But for now, the implementation is only limited to API level and there is no interface for a non-technical person to call these APIs to perform these operations easily. So, for the sake of transparency, we will create an application that serves the purpose of performing these operations by calling these APIs. This work can be considered as the continuous learning of our system to make it as mature as possible.
\subsection{A fused approach of textual and visual information}
During writing the thesis, we came across very interesting research that has been done by one of the researchers in the domain of automatic document classification. Although the targeted type for classification was different, it shares the same nature of the problem. Their approach towards the problem was to use the combined information of text and visual parts to solve the problem. The idea of this model is to embed visually distinguishable information against the important information which may be visually indistinguishable to the documents based on the content extracted from the document to help the model to classify the documents easily. It is an obvious fact that for a machine learning model, it is easier to classify visual information rather than textual information in the form of an image. We can apply the same approach to our problem as well to make more accurate predictions and good results. So, in the future, we can also plan to revamp little bit our existing solution with this approach and see how it goes. As far as I think, it will definitely add up more value to the existing solution.