\par
% In the final chapter, we will briefly summarize the goals, implementations, and limitation of the thesis. Also, we will describe the ideas for future work and what are the possible ways to enhance the solution and achieve maximum results.
\section{Conclusion}
The main objective of this thesis, the classification of lanes in traffic scene using Convolutional Neural Network has been achieved. For this purpose, LeNet-5 machine learning architecture was used to train the classification model using screenshots of traffic scene captured from our simulation. The synthetic images are gathered by our computer simulation which we also developed using Unity3D. All the images were pre-processed and normalized before being fed to the neural network. We have used Sofmax cross-entropy as the loss function and the model was optimized using Gradient Descent algorithm with a learning rate of 0.01 and the batch size of 128. After training for 10 epochs on 14,547,016 samples we reached the loss of \textit{0.025985} and a very good test accuracy of \textit{0.9860} for 3,636,754 samples.
We have demonstrated that using a visualization. The visualization consist of a traffic scene image with predicted pixel is color coded and shown over the image at the respective coordinate. From this work we conclude that it is possible to automatically classify lanes based on inexpensive high-level traffic data from computer simulation. The incorrect predictions were mostly because of occlusion of lane by cars. We also conclude that the our network is pre-trained on synthetic data which can be fine-tuned with real data later. However, we cannot use this model with 

\section{Future Work}
The results of this thesis have been very interesting and can be topics for further study and researches. Here, we will discuss the options for improving or enhancing the approach offered, what can be the possible efforts to achieve this and what are the things required to do in the future. For now, we have defined few actions to take after the study in the future.
\par
As we mentioned in Chapter \ref{chap:3}, the data gathered are with ego-vehicle in the rightmost lane of the roads. So the first thing we need to do is run the simulation with ego-vehicle in different lanes. Further work on data gathering of complex traffic scenes would also help us in developing a robust traffic scene classifier. That is why situations which are complex, rarely happen or are dangerous to perform in real-world should be modeled in the computer simulation. For example urban environments, traffic congestion, closures, crashes and accidents.
\par
To further our research we plan use modern Convolutional Networks like AlexNet, VGGNet, GoogLeNet and ResNet. These neural networks were at one point winners or runners-up in the famous ImageNet competition, which has served as a barometer of progress on supervised learning in computer vision since 2010. The ImageNet project is a large visual database designed for use in research of visual object recognition software. The ImageNet project runs an annual software contest, the ImageNet Large Scale Visual Recognition Challenge, where software programs participate in classifying and identifying scenes and objects correctly. These neural networks are designed to identify visual patterns directly from pixel images with minimal preprocessing.

\par
Further work needs to be performed to establish whether the classifier is actually learning from road features or or have simply learned the postions of lanes. Our results are encouraging and should be validated by a larger sample of complex traffic scenarios.

% receptive field
% complex scenario => close to real world data=> and test with real data
% - [ ] Add complex scenarios, currently on Ego-lane
% - [ ] We can mix Hough transform results and CNN results 
% - transferlearning,
% - we can try instead of putting pixel coordinates in the image we can 
% lane marking and road signs